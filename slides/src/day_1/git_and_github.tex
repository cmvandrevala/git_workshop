% ======================================= %
% GIT AND GITHUB
% ======================================= %

\section[GitHub]{Combining Git With GitHub}

\begin{frame}
\frametitle{Remote Repositories}
\begin{itemize}
\item Part of the strength of Git is linking your repository with other remote repositories
\item We will mostly talk about this in the context of Git and GitHub
\end{itemize}
\end{frame}
\note{}

\begin{frame}
\frametitle{View All Remote Repositories}
In order to view all of the remote repositories for a project use:
\begin{itemize}
\item \emph{git remote -v}
\end{itemize}
\end{frame}
\note{}

\begin{frame}
\frametitle{Add Remote Repositories}
In order to add a remote repository use:
\begin{itemize}
\item \emph{git remote add some\_url}
\end{itemize}
\end{frame}
\note{}

\begin{frame}
\frametitle{Pull From Remote Repositories}
In order to pull data from a remote repository use:
\begin{center}
\emph{git fetch some\_url}\\
\emph{git -a commit}
\end{center}
\end{frame}
\note{}

\begin{frame}
\frametitle{Push to Remote Repositories}
In order to push data to a remote repository use:
\begin{center}
\emph{git push remote\_name branch\_name}
\end{center}
\end{frame}
\note{}

\begin{frame}
\frametitle{GitHub Demo}
\begin{center}
It's easier to demonstrate this than to write slides about it...
\end{center}
\end{frame}
\note{}