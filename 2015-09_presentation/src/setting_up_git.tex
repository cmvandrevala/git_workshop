% ======================================= %
% SETTING UP GIT
% ======================================= %

\section[Setting Up Git]{Setting Up Git On Your Machine}

\begin{frame}
\frametitle{\large{Setting Up Git - Linux}}
You can use the package management tool that comes with your distribution (use sudo):
\begin{enumerate}
\item yum install git
\item apt-get install git
\end{enumerate}
\end{frame}
\note{}

\begin{frame}
\frametitle{\large{Setting Up Git - Mac}}
There are three main ways to install Git:
\begin{enumerate}
\item Install the Xcode Command Line Tools and Type ``git'' Into the Terminal
\item Binary Installer: \href{http://git-scm.com/download/mac}{http://git-scm.com/download/mac}
\item Git/GitHub GUI: \href{https://mac.github.com/}{https://mac.github.com/}
\end{enumerate}
\end{frame}
\note[itemize]{
\item The GUI only implements a subset of the full Git functionality, so it is best to learn how to use the command line.
}

\begin{frame}
\frametitle{\large{Setting Up Git - Windows}}
There are three main ways to install Git:
\begin{enumerate}
\item Binary Installer: \href{http://git-scm.com/download/win}{http://git-scm.com/download/win}
\item msysGit: \href{http://msysgit.github.io/}{http://msysgit.github.io/}
\item Git/GitHub GUI: \href{https://windows.github.com/}{https://windows.github.com/}
\end{enumerate}
\end{frame}
\note[itemize]{
\item The GUI only implements a subset of the full Git functionality, so it is best to learn how to use the command line.
}

\begin{frame}
\frametitle{\large{Setting Up Git - Installing From Source}}
You can also install GitHub from source. See the Git website for full instructions on how to do that.
\end{frame}
\note[itemize]{
\item http://git-scm.com/
}

\begin{frame}
\frametitle{\large{Setting Up Git - Config File}}
Git stores user information in \emph{/etc/gitconfig}, \emph{~/.gitconfig}, and \emph{/your-project/.git/config}. To set up your information: \\
\begin{itemize}
\item \emph{git config -{}-global user.name ``Cyrus Vandrevala''}
\item \emph{git config -{}-global user.email cvandrev@purdue.edu}
\item \emph{git config -{}-global core.editor vim}
\end{itemize}
\end{frame}
\note{}

\begin{frame}
\frametitle{\large{Setting Up Git - Config File}}
You can double check the information you entered by using: \\
\begin{itemize}
\item \emph{git config -{}-list}
\end{itemize}
\end{frame}
\note{}

\begin{frame}
\frametitle{\large{Setting Up a New Git Repo}}
\begin{enumerate}
\item Create a New Directory (mkdir my-awesome-directory)
\item Navigate Into the Directory (cd my-awesome-directory)
\item Initialize the Directory (git init)
\end{enumerate}
\vspace{5mm}
The git init command creates a hidden directory called .git that contains all of the metadata for the project. \emph{You should never change anything in .git directly!}
\end{frame}
\note{}

\begin{frame}
\frametitle{\large{Retrieving an Existing Git Repo}}
\begin{enumerate}
\item Navigate to the Directory Where You Want to Store the Project
\item Run \emph{git clone https://mydirectory.com/}
\end{enumerate}
\vspace{5mm}
\begin{itemize}
\item Git supports many transfer protocols (including SSH)
\item Remember, you are creating a standalone copy of the entire project.
\end{itemize}
\end{frame}
\note{}